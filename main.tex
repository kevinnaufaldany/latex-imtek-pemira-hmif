\documentclass[12pt,a4paper]{article}
%%%%%%%%%%%%%%%%%%%%%%%%% Credit %%%%%%%%%%%%%%%%%%%%%%%%

% template ini dibuat oleh kevin.122140222@student.itera.ac.id untuk dipergunakan oleh dipergunakan oleh Pemilu HMIF 2025.
% Template dibuat berdasarkan template LaTeX Template IF ITERA oleh Sivitas Akademik ITERA, khususnya Program Studi Teknik Informatika ITERA.

%%%%%%%%%%%%%%%%%%%%%%%%% PACKAGE starts HERE %%%%%%%%%%%%%%%%%%%%%%%%
\usepackage{graphicx}
\usepackage{caption}
\captionsetup[table]{name=Tabel}
\captionsetup[figure]{name=Gambar}
\usepackage{tabulary}
% \usepackage{amsmath}
\usepackage{fancyhdr}
% \usepackage{amssymb}
% \usepackage{amsthm}
\usepackage{placeins}
\usepackage{float}
% \usepackage{amsfonts}
\usepackage{graphicx}
\usepackage[all]{xy}
\usepackage{tikz}
\usepackage{verbatim}
% \usepackage[left=2cm,right=2cm,top=3cm,bottom=2.5cm]{geometry}
\usepackage[left=2cm,right=2cm,top=5cm,bottom=3cm,headheight=100pt,headsep=1cm,footskip=1.5cm]{geometry}
\usepackage{hyperref}
\hypersetup{
    colorlinks,
    linkcolor=black,
    citecolor={blue!50!black},
    urlcolor={blue!80!black}
}
\usepackage{libertine}
\usepackage{libertinust1math}
\usepackage[T1]{fontenc}
\usepackage{inconsolata}

\usepackage{caption}
\usepackage{subcaption}
\usepackage{multirow}
\usepackage{psfrag}
\usepackage[T1]{fontenc}
\usepackage[scaled]{beramono}
% Enable inserting code into the document
\usepackage{listings}
\usepackage{xcolor}
\usepackage{enumitem} % for custom enumerate/itemize formatting 
% custom color & style for listing
\definecolor{codegreen}{rgb}{0,0.6,0}
\definecolor{codegray}{rgb}{0.5,0.5,0.5}
\definecolor{codepurple}{rgb}{0.58,0,0.82}
\definecolor{backcolour}{rgb}{0.95,0.95,0.92}
\lstdefinestyle{mystyle}{
	backgroundcolor=\color{backcolour},   
	commentstyle=\color{green},
	keywordstyle=\color{codegreen},
	numberstyle=\tiny\color{codegray},
	stringstyle=\color{codepurple},
	basicstyle=\ttfamily\footnotesize,
	breakatwhitespace=false,         
	breaklines=true,                 
	captionpos=b,                    
	keepspaces=true,                 
	numbers=left,                    
	numbersep=5pt,                  
	showspaces=false,                
	showstringspaces=false,
	showtabs=false,                  
	tabsize=2
}
\lstset{style=mystyle}
\renewcommand{\lstlistingname}{Kode}
%%%%%%%%%%%%%%%%%%%%%%%%% PACKAGE ends HERE %%%%%%%%%%%%%%%%%%%%%%%%


%%%%%%%%%%%%%%%%%%%%%%%%% Data Diri %%%%%%%%%%%%%%%%%%%%%%%%
% \newcommand{\stuid}{isi dengan NIM juga}
% \newcommand{\student}{\textbf{Isi Nama Di Sini (\stuid{})}}
% \newcommand{\course}{\textbf{Nama Mata Kuliah (Kode Mata Kuliah)}}
% \newcommand{\assignment}{\textbf{xxx}} % tugas ke...

%%%%%%%%%%%%%%%%%%% using theorem style %%%%%%%%%%%%%%%%%%%%
\newtheorem{thm}{Theorem}
\newtheorem{lem}[thm]{Lemma}
\newtheorem{defn}[thm]{Definition}
\newtheorem{exa}[thm]{Example}
\newtheorem{rem}[thm]{Remark}
\newtheorem{coro}[thm]{Corollary}
\newtheorem{quest}{Question}[section]
%%%%%%%%%%%%%%%%%%%%%%%%%%%%%%%%%%%%%%%%
\usepackage{lipsum}%% a garbage package you don't need except to create examples.
\usepackage{fancyhdr}
\usepackage[ddmmyyyy]{datetime}
\usepackage{array}

\pagestyle{fancy}
\fancyhf{} % clear semua

% ====================================
% HEADER FIX — Dibungkus parbox
% ====================================
\fancyhead[C]{
\parbox{\textwidth}{
\centering
\begin{tabular}{m{1.7cm} m{11.8cm} m{2.2cm}}
\includegraphics[width=1.7cm]{Figure/ITERA.png} &

\parbox[c]{11.8cm}{
    \centering
    {\large \textbf{HIMPUNAN MAHASISWA INFORMATIKA (HMIF)}}\\
    {\large \textbf{FAKULTAS TEKNOLOGI INDUSTRI}}\\
    {\large \textbf{INSTITUT TEKNOLOGI SUMATERA}}\\
    \footnotesize \textit{Jl. Terusan Ryacudu, Way Huwi, Jati Agung, Lampung Selatan 35365}\\
    \footnotesize \textit{e-mail: \href{mailto:hmifitera1@gmail.com}{hmifitera1@gmail.com}}
}
&

\includegraphics[width=2.2cm]{Figure/HMIF.png}
\end{tabular}

% garis bawah header
\vspace{0.1cm}
\noindent\rule{\textwidth}{0.4pt}
}}

% untuk ganti nama dokumen
\newcommand{\dokumen}{GUIDE BOOK \\[0.5cm] PEMILIHAN}
\newcommand{\dokumenkecil}{Guide Book Pemilihan }
\newcommand{\pemilu}{PEMILU }
\newcommand{\pemilukecil}{Pemilu }
\newcommand{\tahun}{HMIF ITERA 2025}


% \fancyfoot[L]{\textbf{Manual Book}}
\fancyfoot[L]{\textbf{\dokumenkecil \pemilu \tahun \hfill \thepage}}

\renewcommand{\headrulewidth}{0.4pt}
\renewcommand{\footrulewidth}{0.4pt}

\usepackage{tocloft} % untuk formatting daftar isi

% Konfigurasi Daftar Isi
\renewcommand{\contentsname}{DAFTAR ISI}
\renewcommand{\cfttoctitlefont}{\hfill\Large\bfseries}
\renewcommand{\cftaftertoctitle}{\hfill}
\setlength{\cftbeforesecskip}{6pt}
\renewcommand{\cftsecleader}{\cftdotfill{\cftdotsep}}
\renewcommand{\cftsecfont}{\bfseries}
\renewcommand{\cftsecpagefont}{\bfseries}
\cftsetindents{section}{0em}{2.3em}
\cftsetindents{subsection}{2.3em}{3em}


%%%%%%%%%%%%%%  Shortcut for usual set of numbers  %%%%%%%%%%%

\newcommand{\N}{\mathbb{N}}
\newcommand{\Z}{\mathbb{Z}}
\newcommand{\Q}{\mathbb{Q}}
\newcommand{\R}{\mathbb{R}}
\newcommand{\C}{\mathbb{C}}
% \setlength\headheight{14pt}

%%%%%%%%%%%%%%%%%%%%%%%%%%%%%%%%%%%%%%%%%%%%%%%%%%%%%%%555

\begin{document}

% Gunakan penomoran romawi untuk cover dan daftar isi
\pagenumbering{roman}
\thispagestyle{empty}

% ===== COVER PAGE =====
\begin{center}
    \vspace*{-1cm}
    {\Huge \textbf{\dokumen}} \\[2cm]
    
    % Logo HMIF di tengah
    \includegraphics[width=10cm]{Figure/HMIF.png} \\[2cm]
    
    {\Large Disusun oleh:} \\[0.5cm]
    {\LARGE \textbf{DIVISI IMTEK}} \\[1.5cm]
    
    {\LARGE \textbf{\pemilu HMIF}} \\[0.5cm]
    {\LARGE \textbf{2025}}
\end{center}


% ==========================
%  DAFTAR ISI
% ==========================
\newpage
\vspace*{-2.5cm}
\renewcommand{\contentsname}{DAFTAR ISI}
\tableofcontents
\clearpage

%%%%%%%%%%%%%%%%%%%%%%%%%%%%%%%%%%%%%%%%%%%%% BODY DOCUMENT %%%%%%%%%%%%%%%%%%%%%%%%%%%%%%%%%%%%%%%%%%%%%
% Mulai penomoran halaman arabic dari sini
\pagenumbering{arabic}
\pagestyle{fancy}

\section{Persiapan}
    Persiapan yang diperlukan berupa:
    \begin{enumerate}[label=\arabic*., labelindent=0.75cm, leftmargin=!, itemindent=0pt]
        \item Device Handphone 
        \item Internet 
        \item Browser dengan \textit{e-mail} ITERA
        \item KTM / e-KTM 
    \end{enumerate}

\subsection{Masuk ke Website}
Silahkan masuk ke website dengan cara scan barcode yang di siapkan oleh panitia, atau dapat di akses pada url berikut \href{https://pemilu-hmif-voting.vercel.app/}{https://pemilu-hmif-voting.vercel.app/}

\subsection{Login}
Silahkan melakukan \textbf{Login} menggunakan akun \textit{e-mail} ITERA

\begin{figure}[h]
	\centering
	\begin{subfigure}[b]{0.4\textwidth}
		\centering
		\def\svgwidth{\columnwidth}
		\includegraphics[width=1\textwidth]{Figure/2.Tampilan Awal.jpeg}
		\caption{Click \textbf{Masuk dengan akun ITERA}}
		\label{fig:tampilan-awal}
	\end{subfigure}
	\qquad %add desired spacing between images, e. g. ~, \quad, \qquad, \hfill etc. 
	%(or a blank line to force the subfigure onto a new line)
	\begin{subfigure}[b]{0.4\textwidth}
		\centering
		\def\svgwidth{\columnwidth}
		\includegraphics[width=1\textwidth]{Figure/6. Berhasil Login.jpeg}
		\caption{Tampilan Awal Setelah \textbf{Login}}
		\label{fig:berhasil-login}
	\end{subfigure}
	\caption{Masuk ke Website dan Login}\label{fig:masuk-website}
\end{figure}

Pastikan status pada sistem masih menunjukkan \textbf{Belum Tervalidasi}. Sebelum memasuki ruangan pemilihan, pastikan Anda membawa dan menyiapkan KTM 
(\textit{e-KTM}). 

\subsection{Baca Visi dan Misi}
Baca visi dan misi dari setiap kandidat yang tersedia sebelum melakukan pemilihan

\begin{figure}[h]
	\centering
	% Baris pertama: dua gambar sejajar
	\begin{subfigure}[b]{0.4\textwidth}
		\centering
		\includegraphics[width=1\textwidth]{Figure/7. Lihat visi misi calon kahim.jpeg}
		\caption{Click \textbf{Lihat Visi \& Misi}}
		\label{fig:lihat-visimisi}
	\end{subfigure}
	\hspace{0.5cm}
	\begin{subfigure}[b]{0.4\textwidth}
		\centering
		\includegraphics[width=1\textwidth]{Figure/8. Visi misi calon kahim.jpeg}
		\caption{Bisa di baca isi dari visi \& misi kandidat}
		\label{fig:isi-visimisi2}
	\end{subfigure}

	\caption{Baca Visi \& Misi}
	\label{fig:baca-visimisi}
\end{figure}

\section{Validasi}
Sebelum melanjutkan proses validasi, pastikan kembali bahwa KTM (\textit{e-KTM}) telah disiapkan. Lakukan proses \textbf{validasi} kepada petugas validator yang berada pada kursi yang telah ditentukan. Ikuti seluruh instruksi proses \textbf{validasi} yang disampaikan oleh petugas validator. Jika sudah, Lakukan refresh halaman hingga status berubah menjadi \textbf{Tervalidasi}.

\begin{figure}[H]
	\centering
	% Baris pertama: dua gambar sejajar
	\begin{subfigure}[b]{0.4\textwidth}
		\centering
		\includegraphics[width=1\textwidth]{Figure/12. Refresh.jpeg}
		\caption{Click \textbf{Refresh}}
		\label{fig:refresh}
	\end{subfigure}
	\hspace{0.5cm}
	\begin{subfigure}[b]{0.4\textwidth}
		\centering
		\includegraphics[width=1\textwidth]{Figure/13. sUDAH tERVALIDASI oLEH PANITIA.jpeg}
		\caption{Status \textbf{Tervalidasi}}
		\label{fig:status-tervalidasi}
	\end{subfigure}

	\caption{Validasi}
	\label{fig:validasi}
\end{figure}

Setelah proses validasi dinyatakan berhasil, Anda dapat melanjutkan ke tahap pemilihan. \\ 
\textbf{Catatan:} Pemilihan hanya dapat dilakukan pada kursi yang telah disediakan di depan validator.

\section{Pemilihan}
Setelah status berubah menjadi \textbf{Tervalidasi}, Anda dapat melanjutkan ke tahap pemilihan dengan cara menekan tombol \textbf{Pilih} pada setiap kandidat, kemudian pastikan Anda sudah memilih \textbf{Calon Ketua Himpunan} dan \textbf{Calon Senator}.
\subsection{Proses Pemilihan}

\begin{figure}[H]
	\centering
		% Baris pertama: dua gambar sejajar
		\begin{subfigure}[b]{0.3\textwidth}
			\centering
			\includegraphics[width=\textwidth]{Figure/14. Pilih kahim.jpeg}
			\caption{Click \textbf{Pilih} pada kandidat yang ingin dipilih}
			\label{fig:3.CitraAwalFlip}
		\end{subfigure}
		
		% Baris kedua: satu gambar di bawah tengah
		\vskip\baselineskip
		\begin{subfigure}[b]{0.35\textwidth}
			\centering
			\includegraphics[width=1\textwidth]{Figure/20. Klik Simpan kedua pilihan.jpeg}
			\caption{Click \textbf{Simpan kedua pilihan} setelah memilih kedua kandidat}
			\label{fig:3.hasilFlipHorizontal}
		\end{subfigure}
		\hspace{0.7cm}
		\begin{subfigure}[b]{0.35\textwidth}
			\centering
			\includegraphics[width=1\textwidth]{Figure/21. Konfirmasi pilihan.jpeg}
			\caption{Click \textbf{Ya} pada pilihan konfirmasi untuk menyelesaikan pemilihan}
			\label{fig:3.hasilFlipVertical}
		\end{subfigure}
		\caption{Proses Pemilihan}
		\label{fig:3.HasilFlipData}
\end{figure}

\subsection{Sudah Memilih}
Jikalau sudah memilih, Pastikan status sistem berubah menjadi \textbf{Sudah Memilih}, dan jikalau sudah memilih dan status sudah berubah menjadi \textbf{Sudah Memilih}, lakukan \textbf{validasi} akhir kepada petugas validator sebagai syarat untuk meninggalkan ruangan.

% 22. Berhasil Memilih.jpeg
\begin{figure}[H] % Kalau menggunakan H, posisi gambar akan tepat dibawah teks
    \centering
    \includegraphics[width=0.5\textwidth]{Figure/22. Berhasil Memilih.jpeg}
    \caption{Contoh Gambar Dataset}
    \label{fig:3.ContohDataset}
\end{figure}

\newpage
\section{Log out}
Setelah proses pemilihan dan validasi akhir selesai, silahkan click icon pintu di pojok kanan bawah berwarna merah untuk melakukan \textbf{Log out}.

\begin{figure}[H]
	\centering
	% Baris pertama: dua gambar sejajar
	\begin{subfigure}[b]{0.4\textwidth}
		\centering
		\includegraphics[width=1\textwidth]{Figure/logout.jpeg}
		\caption{Click icon \textbf{pintu} untuk log out}
		\label{fig:logout-icon}
	\end{subfigure}
	\hspace{0.5cm}
	\begin{subfigure}[b]{0.4\textwidth}
		\centering
		\includegraphics[width=1\textwidth]{Figure/2.Tampilan Awal.jpeg}
		\caption{Balik ke tampilan awal setelah log out}
		\label{fig:balik-tampilan-awal}
	\end{subfigure}

	\caption{Log out}
	\label{fig:logout}
\end{figure}

\end{document}