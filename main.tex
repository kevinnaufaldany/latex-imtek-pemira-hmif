\documentclass[11pt,a4paper]{article}
%%%%%%%%%%%%%%%%%%%%%%%%% Credit %%%%%%%%%%%%%%%%%%%%%%%%

% template ini dibuat oleh kevin.122140222@student.itera.ac.id untuk dipergunakan oleh dipergunakan oleh Pemilu HMIF 2025.
% Template dibuat berdasarkan template LaTeX Template IF ITERA oleh Sivitas Akademik ITERA, khususnya Program Studi Teknik Informatika ITERA.

%%%%%%%%%%%%%%%%%%%%%%%%% PACKAGE starts HERE %%%%%%%%%%%%%%%%%%%%%%%%
\usepackage{graphicx}
\usepackage{caption}
\captionsetup[table]{name=Tabel}
\captionsetup[figure]{name=Gambar}
\usepackage{tabulary}
% \usepackage{amsmath}
\usepackage{fancyhdr}
% \usepackage{amssymb}
% \usepackage{amsthm}
\usepackage{placeins}
% \usepackage{amsfonts}
\usepackage{graphicx}
\usepackage[all]{xy}
\usepackage{tikz}
\usepackage{verbatim}
% \usepackage[left=2cm,right=2cm,top=3cm,bottom=2.5cm]{geometry}
\usepackage[left=2cm,right=2cm,top=5cm,bottom=3cm,headheight=100pt,headsep=1cm,footskip=1.5cm]{geometry}
\usepackage{hyperref}
\hypersetup{
    colorlinks,
    linkcolor=black,
    citecolor={blue!50!black},
    urlcolor={blue!80!black}
}
\usepackage{libertine}
\usepackage{libertinust1math}
\usepackage[T1]{fontenc}
\usepackage{inconsolata}

\usepackage{caption}
\usepackage{subcaption}
\usepackage{multirow}
\usepackage{psfrag}
\usepackage[T1]{fontenc}
\usepackage[scaled]{beramono}
% Enable inserting code into the document
\usepackage{listings}
\usepackage{xcolor} 
% custom color & style for listing
\definecolor{codegreen}{rgb}{0,0.6,0}
\definecolor{codegray}{rgb}{0.5,0.5,0.5}
\definecolor{codepurple}{rgb}{0.58,0,0.82}
\definecolor{backcolour}{rgb}{0.95,0.95,0.92}
\lstdefinestyle{mystyle}{
	backgroundcolor=\color{backcolour},   
	commentstyle=\color{green},
	keywordstyle=\color{codegreen},
	numberstyle=\tiny\color{codegray},
	stringstyle=\color{codepurple},
	basicstyle=\ttfamily\footnotesize,
	breakatwhitespace=false,         
	breaklines=true,                 
	captionpos=b,                    
	keepspaces=true,                 
	numbers=left,                    
	numbersep=5pt,                  
	showspaces=false,                
	showstringspaces=false,
	showtabs=false,                  
	tabsize=2
}
\lstset{style=mystyle}
\renewcommand{\lstlistingname}{Kode}
%%%%%%%%%%%%%%%%%%%%%%%%% PACKAGE ends HERE %%%%%%%%%%%%%%%%%%%%%%%%
%===========================
% Custom counter for Pasal
%===========================
\newcounter{pasal}

% Command Pasal otomatis dengan argumen judul
\newcommand{\pasal}[1]{%
    \stepcounter{pasal}%
    \addcontentsline{toc}{section}{Pasal \thepasal}
    \vspace{0.5em}
    \begin{center}
    \Large \textbf{Pasal \thepasal}\\
    \vspace{0.2em}
    \textbf{#1}
    \end{center}
    \vspace{0.2em}
}

\newcounter{subbagian}[pasal]
\renewcommand{\thesubbagian}{\alph{subbagian}}

\newcommand{\subbagian}{%
    \stepcounter{subbagian}%
    \textbf{\thesubbagian.\ }%
}

% Mengurangi jarak besar sebelum/after section
\usepackage{titlesec}
\titlespacing*{\section}{0pt}{0.5em}{0.3em}

% Mengurangi jarak antar item enumerate
\usepackage{enumitem}
\setlist{nosep, leftmargin=1.2cm}

\usepackage{ulem}


%%%%%%%%%%%%%%%%%%%%%%%%% Data Diri %%%%%%%%%%%%%%%%%%%%%%%%
% \newcommand{\stuid}{isi dengan NIM juga}
% \newcommand{\student}{\textbf{Isi Nama Di Sini (\stuid{})}}
% \newcommand{\course}{\textbf{Nama Mata Kuliah (Kode Mata Kuliah)}}
% \newcommand{\assignment}{\textbf{xxx}} % tugas ke...

%%%%%%%%%%%%%%%%%%% using theorem style %%%%%%%%%%%%%%%%%%%%
\newtheorem{thm}{Theorem}
\newtheorem{lem}[thm]{Lemma}
\newtheorem{defn}[thm]{Definition}
\newtheorem{exa}[thm]{Example}
\newtheorem{rem}[thm]{Remark}
\newtheorem{coro}[thm]{Corollary}
\newtheorem{quest}{Question}[section]
%%%%%%%%%%%%%%%%%%%%%%%%%%%%%%%%%%%%%%%%
\usepackage{lipsum}%% a garbage package you don't need except to create examples.
\usepackage{fancyhdr}
\usepackage[ddmmyyyy]{datetime}
\usepackage{array}
\usepackage[bahasa]{babel}

\pagestyle{fancy}
\fancyhf{} % clear semua

% ====================================
% HEADER FIX — Dibungkus parbox
% ====================================
\fancyhead[C]{
\parbox{\textwidth}{
\centering
\begin{tabular}{m{2.2cm} m{10.8cm} m{2.7cm}}
\includegraphics[width=2.2cm]{Figure/ITERA.png} &

\parbox[c]{10.8cm}{
    \centering
    {\large \textbf{HIMPUNAN MAHASISWA INFORMATIKA (HMIF)}}\\
    {\large \textbf{FAKULTAS TEKNOLOGI INDUSTRI}}\\
    {\large \textbf{INSTITUT TEKNOLOGI SUMATERA}}\\
    \footnotesize \textit{Jl. Terusan Ryacudu, Way Huwi, Jati Agung, Lampung Selatan 35365}\\
    \footnotesize \textit{e-mail: \href{mailto:hmifitera1@gmail.com}{hmifitera1@gmail.com}}
}
&

\includegraphics[width=2.7cm]{Figure/HMIF.png}
\end{tabular}

% garis bawah header
\vspace{0.1cm}
\noindent\rule{\textwidth}{0.4pt}
}}

% untuk ganti nama dokumen
\newcommand{\dokumen}{SOP PESERTA }
\newcommand{\dokumenkecil}{SOP Peserta }
\newcommand{\pemilu}{PEMILU }
\newcommand{\pemilukecil}{Pemilu }


% \fancyfoot[L]{\textbf{Manual Book}}
\fancyfoot[L]{\textbf{\dokumenkecil by Imtek \pemilukecil HMIF 2025 \hfill \thepage}}

\renewcommand{\headrulewidth}{0.4pt}
\renewcommand{\footrulewidth}{0.4pt}

\usepackage{tocloft} % untuk formatting daftar isi

% Konfigurasi Daftar Isi
\renewcommand{\contentsname}{DAFTAR ISI}
\renewcommand{\cfttoctitlefont}{\hfill\Large\bfseries}
\renewcommand{\cftaftertoctitle}{\hfill}
\setlength{\cftbeforesecskip}{6pt}
\renewcommand{\cftsecleader}{\cftdotfill{\cftdotsep}}
\renewcommand{\cftsecfont}{\bfseries}
\renewcommand{\cftsecpagefont}{\bfseries}
\cftsetindents{section}{0em}{2.3em}
\cftsetindents{subsection}{2.3em}{3em}


%%%%%%%%%%%%%%  Shortcut for usual set of numbers  %%%%%%%%%%%

\newcommand{\N}{\mathbb{N}}
\newcommand{\Z}{\mathbb{Z}}
\newcommand{\Q}{\mathbb{Q}}
\newcommand{\R}{\mathbb{R}}
\newcommand{\C}{\mathbb{C}}
% \setlength\headheight{14pt}

%%%%%%%%%%%%%%%%%%%%%%%%%%%%%%%%%%%%%%%%%%%%%%%%%%%%%%%555

\begin{document}

% Gunakan penomoran romawi untuk cover dan daftar isi
\pagenumbering{roman}
\thispagestyle{empty}

% ===== COVER PAGE =====
\begin{center}
    \vspace*{-1cm}
    {\Huge \textbf{SOP PESERTA}} \\[2cm]
    
    % Logo HMIF di tengah
    \includegraphics[width=10cm]{Figure/HMIF.png} \\[2cm]
    
    {\Large Disusun oleh:} \\[0.5cm]
    {\LARGE \textbf{DIVISI IMTEK}} \\[1.5cm]
    
    {\LARGE \textbf{\pemilu HMIF}} \\[0.5cm]
    {\LARGE \textbf{2025}}
\end{center}


% ==========================
%  DAFTAR ISI
% ==========================
\newpage
% \vspace*{-2.5cm}
% \renewcommand{\contentsname}{DAFTAR ISI}
% \tableofcontents
% \clearpage

%%%%%%%%%%%%%%%%%%%%%%%%%%%%%%%%%%%%%%%%%%%%% BODY DOCUMENT %%%%%%%%%%%%%%%%%%%%%%%%%%%%%%%%%%%%%%%%%%%%%
% Mulai penomoran halaman arabic dari sini
\pagenumbering{arabic}
\pagestyle{fancy}
%========= Pasal 1 ==========
\pasal{Tata Tertib}

\begin{enumerate}
    \item Hadhirin wajib mengenakan pakaian yang rapi dan sopan, serta tidak diperbolehkan mengenakan celana pendek ataupun celana yang memiliki model sobek-sobek \textit{(ripped jeans)} selama acara berlangsung.
    \item Dilarang mengucapkan kata-kata kasar atau melakukan tindakan yang dapat mengganggu kenyamanan hadirin lainnya dalam forum.
    \item Hadhirin tidak diperbolehkan membuat kegaduhan dan wajib mengondisikan alat komunikasi (gawai) dalam mode diam \textit{(silent)} selama kegiatan berlangsung.
    \item Pengambilan dokumentasi selama kegiatan diperbolehkan selama tidak mengganggu pandangan hadirin lain dan jalannya acara.
\end{enumerate}

%========= Pasal 2 ==========
\pasal{Hak dan Kewajiban}

\subbagian \textbf{Kewajiban Hadhirin}
\begin{enumerate}[label=\arabic*., labelindent=0.75cm, leftmargin=!, itemindent=0pt]
    \item Hadhirin berhak mendapatkan informasi yang jelas dan akurat dari narasumber mengenai topik yang dibahas dalam sarasehan.
    \item Hadhirin berhak mengajukan pertanyaan atau tanggapan terkait topik yang dibahas dalam sarasehan.
\end{enumerate}

\textbf{b. Kewajiban Hadhirin}
\begin{enumerate}[label=\arabic*., labelindent=0.75cm, leftmargin=!, itemindent=0pt]
    \item Hadhirin berhak mendapatkan informasi yang jelas dan akurat dari narasumber mengenai topik yang dibahas dalam sarasehan.
    \item Hadhirin berhak mengajukan pertanyaan atau tanggapan terkait topik yang dibahas dalam sarasehan.
\end{enumerate}

%========= Pasal 1 ==========
\pasal{Tata}

\begin{enumerate}
    \item Hadhirin wajib mengenakan pakaian yang rapi dan sopan, serta tidak diperbolehkan mengenakan celana pendek ataupun celana yang memiliki model sobek-sobek \textit{(ripped jeans)} selama acara berlangsung.
    \item Dilarang mengucapkan kata-kata kasar atau melakukan tindakan yang dapat mengganggu kenyamanan hadirin lainnya dalam forum.
    \item Hadhirin tidak diperbolehkan membuat kegaduhan dan wajib mengondisikan alat komunikasi (gawai) dalam mode diam \textit{(silent)} selama kegiatan berlangsung.
    \item Pengambilan dokumentasi selama kegiatan diperbolehkan selama tidak mengganggu pandangan hadirin lain dan jalannya acara.
    \item Dilarang mengucapkan kata-kata kasar atau melakukan tindakan yang dapat mengganggu kenyamanan hadirin lainnya dalam forum.
    \item Hadhirin tidak diperbolehkan membuat kegaduhan dan wajib mengondisikan alat komunikasi (gawai) dalam mode diam \textit{(silent)} selama kegiatan berlangsung.
    \item Pengambilan dokumentasi selama kegiatan diperbolehkan selama tidak mengganggu pandangan hadirin lain dan jalannya acara.
\end{enumerate}

%========= Pasal 1 ==========
\pasal{Tertib}

\begin{enumerate}
    \item Hadhirin wajib mengenakan pakaian yang rapi dan sopan, serta tidak diperbolehkan mengenakan celana pendek ataupun celana yang memiliki model sobek-sobek \textit{(ripped jeans)} selama acara berlangsung.
    \item Dilarang mengucapkan kata-kata kasar atau melakukan tindakan yang dapat mengganggu kenyamanan hadirin lainnya dalam forum.
    \item Hadhirin tidak diperbolehkan membuat kegaduhan dan wajib mengondisikan alat komunikasi (gawai) dalam mode diam \textit{(silent)} selama kegiatan berlangsung.
    \item Pengambilan dokumentasi selama kegiatan diperbolehkan selama tidak mengganggu pandangan hadirin lain dan jalannya acara.
\end{enumerate}

\vspace{1cm}
\begin{flushright}
Lampung Selatan, \today\\
Kepala Divisi Imtek\\[0.2cm]
\includegraphics[width=3cm]{Figure/ttd_kevin.png}\\[0.2cm]
\uline{\textbf{Kevin Naufal Dany}}\\
NIM. 122140222
\end{flushright}

\end{document}